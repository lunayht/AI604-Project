% \documentclass[review]{cvpr}
\documentclass[final]{cvpr}

\usepackage{times}
\usepackage{epsfig}
\usepackage{graphicx}
\usepackage{amsmath}
\usepackage{amssymb}

\newcommand{\todoc}[2]{{\textcolor{#1} {\textbf{[#2]}}}}
\newcommand{\todored}[1]{\todoc{red}{\textbf{#1}}}
\newcommand{\TA}[1]{\todored{TA: #1}}


% Include other packages here, before hyperref.

% If you comment hyperref and then uncomment it, you should delete
% egpaper.aux before re-running latex.  (Or just hit 'q' on the first latex
% run, let it finish, and you should be clear).
\usepackage[pagebackref=true,breaklinks=true,colorlinks,bookmarks=false]{hyperref}


\def\cvprPaperID{****} % *** Enter the CVPR Paper ID here
\def\confYear{CVPR 2021}
% \setcounter{page}{4321} % For final version only


\begin{document}

%%%%%%%%% TITLE
\title{AI604 term project review for submission \TA{Paper ID})}

\author{Reviewer ID: \TA{Reviewer ID}
}

\maketitle

\section{Guideline: Review}

As we announced before, all students should review other teams' papers. For convenience, we have prepared three questions, and you can write reviews by answering them. Each student will review three papers, and \textbf{each review must be submitted in separate files}. \TA{Remove this guideline.} 

\section{Questions}

We expect insightful reviews! Keep in mind that \textbf{the authors of the paper will evaluate your review, and this will be reflected in your final score.}

\subsection{Please provide the summary of the paper (less than 100 words).}

\TA{summary...}

\subsection{Suggestions for the paper (less than 150 words).}

\TA{suggestion...}

\subsection{Final score and detailed comments (less than 150 words).}

\TA{detailed comments...}

% 각 리뷰어는 본인이 리뷰하는 세 논문에 대해 총 12점을 주어야 합니다. 예를 들어 두 개의 논문에 각각 5점을 주었다면, 남은 하나의 논문에는 2점을 주어야 합니다. 이는 어뷰징 (e.g., 논문 모두에 최저점을 주는 행위) 을 방지하기 위한 것으로, 조교가 꼼꼼하게 확인할 예정입니다.
\textbf{IMPORTANT RULE:} You should give a total of 12 points for the three papers you review. For example, if you gave five points to two papers, you should give two points to the remaining one. This rule is a way to prevent abusing (e.g., giving the lowest point for all papers), and TAs will check it strictly.

\TA{Final score: (1/2/3/4/5). -- higher is better}

\section{Etc}

If you have a question, send us an e-mail or a message to the Kakao open chat room.

% {\small
% \bibliographystyle{ieee_fullname}
% \bibliography{reference}
% }

\end{document}
